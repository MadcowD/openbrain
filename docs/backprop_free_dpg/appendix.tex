\section{Appendix}
\textbf{Proof of \eqref{eq:sub_env_com}}
\begin{proof}
  If $E$ and $\scriptn$ are given, then for every $n \in \scriptn$ let $E^n$ be a sub-environment of $E$ with respect to $\scriptn$. Next let $\mu^n$ be a neuromorphically local agent in $E^n$ with respect to $\scriptn$.

  We first show that \eqref{eq:sub_env_com} commutes. Let $v_t \in \scriptv$ and $s_t \in \scripts$. Observe that
\begin{equation*}
\left[\mu \circ \pi_2\right](v_t, s_t) = \mu(s_t) = \delta\left(D(v_t, \epsilon(s_t))\right) = \left[\delta \circ D \circ (\pi_1 \times \epsilon\circ \pi_2)\right](v_t, s_t),
\end{equation*}
and therefore the top half of the diagram commutes. Given some $(v_t, \epsilon(s_t)) \in \scriptv \times \scriptv$ we have that 
\begin{equation*}
  \begin{aligned}
    \left[ \mu^n \circ \pi_1 + \pi_2\right] (v_t, \epsilon(s_t))  &= \mu^n(v_t) + \pi_n(\epsilon(s_t)),\\
     &= \pi_n\left(D(v_t) + \epsilon(s_t)\right), \\
     &= \left[\pi_n\circ D\right](v_t, \epsilon(s_t)). 
  \end{aligned}
\end{equation*}
Thus the diagram in \eqref{eq:sub_env_com} commutes. 
\end{proof}

\textbf{Proof of \eqref{eq:ncompagrees}}
\begin{proof}
  Let $\scriptn, E$ be given and fix $(E^n, \mu^n) \in \mathfrak{D}_\scriptn$.
  For any initial state $s_0$, Theorem  \ref{def:subenv} gives that the state-action trajectory $\kappa \in \Gamma_\mu(\scripts)$ generated by $\mu$ from $s_0$ is dual to the state-action trajectory $\kappa^* \in \Gamma_{\mu^n}(\scriptv)$ generated by $\mu^n$ from $v_0 = 0$ when the hidden state of $T^n$ is $s_t$. Because $E^n$ is a sub-environment of $E$, $r^n \equiv r$ the sequence of rewards on $\kappa$ and $\kappa^*$ are the same. Using the definition of the action-value function assuming that $s_0$ fixed, 
  \begin{equation}\label{eq:qequality}
      Q^{\mu}(s_t, a_t) = \sum_{\tau =t}^\infty r(s_t, \mu(s_t)) \gamma^{\tau - t} = \sum_{\tau =t}^\infty r^n(v_t, \mu^n(v_t)) \gamma^{\tau - t} = Q^{\mu^n}(v_t, a_t)
  \end{equation}
  where $\kappa = ((s_t, \mu(s_t)))_{t\in\mathbb{N}}$ and $\kappa^* = ((v_t,\mu^n(v_t)))_{t\in\mathbb{N}}.$ 


  
If $(v_t, s_t)$ are give, Theorem \ref{thm:ncomp} states that both $\mu^n$ and $\mu$ commute with the dynamics on $(v_t, \epsilon(s_t))$ (see the middle path in \eqref{eq:sub_env_com}). Thus if the dynamics are parameterized by $K^n \in \scriptk_n$, application of the equality in \eqref{eq:qequality}, gives the following commutitive diagram, $\mathfrak{k}_n$:
  \begin{equation}\label{eq:kparam}
    \begin{tikzcd}
    &[+15pt] \scripta \arrow{rd}{Q^{\mu}(s_t, \cdot)}\\[+18pt]
      \scriptk_n \arrow{r}{D(v_t, \epsilon(s_t))}
            \arrow{rd}[swap]{\mu^n(v_t)} 
            \arrow{ru}{\mu(s_t)} & \scriptv \arrow{d}[swap]{\pi_n}\arrow{u}{\delta} & \mathbb{R}\\[+18pt]
      & \mathbb{R} \arrow{ru}[swap]{Q^{\mu^n}(v_t, \cdot)}
    \end{tikzcd}
  \end{equation}
  Recall from category theory that differentiation is a functor on the category of $C^1$ manifolds $\mathrm{\textbf{Man}}^1$ because for any two
  morphisms $f,g \in \mathrm{Hom}(\mathrm{\textbf{Man}}^1),$ $\nabla f \circ g = \nabla f \circ \nabla g$ by chain rule. It then follows that the diagram $\nabla(\mathfrak{k}_n)$ commutes and so
  \begin{equation*}
  \begin{aligned}
    \nabla_{K^n} Q^{\mu^n}(v_t, \alpha) \Big|_{\alpha=\mu^n(v_t)}
    &= \nabla_\alpha Q^{\mu^n}(v_t, \alpha) \nabla_{K^n} \mu^n(v_t) \\
    &=\nabla_a Q^{\mu}(s_t, a) \nabla_{K^n} \mu(s_t) \\
    &=\nabla_{K^n} Q^\mu(s_t, a) \Big|_{a = \mu(s_t)}
  \end{aligned}
  \end{equation*}
  
  Because this equality holds for any $(s_t, v_t)$ and therefore any state-action trajectory with arbitrary initial hidden variables, the theorem holds.
\end{proof}